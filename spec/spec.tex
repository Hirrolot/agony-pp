\documentclass[12pt]{article}

\usepackage{hyperref}
\usepackage{float}
\usepackage{bussproofs}
\usepackage{minted}
\usepackage{amsmath}
\usepackage{amssymb}
\usepackage{ntheorem}
\usepackage{geometry}
\usepackage{biblatex}
\usepackage{csquotes}
\usepackage{tabularx}
\usepackage[english]{babel}

\geometry{legalpaper}
\theoremstyle{break}
\newtheorem{notation}{Notation}
\newtheorem{example}{Example}

\addbibresource{references.bib}

\floatstyle{boxed}
\restylefloat{figure}

\begin{document}

\title{Macrolop Specification}
\date{\today}
\author{Temirkhan Myrzamadi \\ e-mail: \href{mailto:hirrolot@gmail.com}{hirrolot@gmail.com}}
\maketitle

\begin{abstract}
This paper formally describes the form and execution of metaprograms written in Macrolop,
an embedded metalanguage aimed at language-oriented programming in C. See also the official
repository \cite{Macrolop} for the user-friendly overview and the accompanied standard
library \cite{MacrolopDocs}.
\end{abstract}

\tableofcontents

\newpage

\section{EBNF Grammar}

\begin{figure}[H]
    \caption{Grammar rules}

\begin{minted}{bnf}
<eval> ::= "MACROLOP_EVAL(" { <term> }+ ")" ;

<term> ::= "call(" <op> "," { <term> }* ")"
         | "v(" <preprocessor-token-list> ")" ;

<op>   ::= <ident> | { <term> }+ ;
\end{minted}

\end{figure}

A metaprogram in Macrolop consists of a non-empty sequence of terms, each of which
is either a macro call or just a value.

Notes:

\begin{itemize}
    \item The grammar above describes metaprograms already expanded by the C preprocessor,
    except for \texttt{MACROLOP\_EVAL}, \texttt{call}, and \texttt{v}.
    \item \texttt{call} accepts \texttt{op} either as an identifier or as a non-empty
    sequence of terms that reduces to an identifier.
    \item \texttt{call} accepts arguments without a separator. This is intentional: it lets
    all arguments be generated uniformly, unlike separation by commas where the last argument
    need no comma after itself.
\end{itemize}

However, the \texttt{call} syntax hurts IDE support: bad code formatting, no parameters
documentation highlighting, et cetera. The workaround is to define a wrapper around
an implementation macro like this:

\begin{minted}{c}
/// A documentation string here.
#define FOO(a, b, c) FOO_REAL(a b c)
#define FOO_REAL(a, b, c) // The actual implementation here.
\end{minted}

Then \texttt{FOO} can be called as \texttt{FOO(v(1), v(2), v(3))}.

All the public std's macros follow this convention, and moreover, std's public
higher-order macros require so for supplied user macros.

\section{Reduction Semantics}

We define reduction semantics for Macrolop. The abstract machine executes configurations
of the form $\langle K; A; C \rangle$:

\begin{itemize}
    \item $K$ is a continuation of the form $\langle K; A; C \rangle$, where
    $C$ includes the $?$ sign denoting a result passed into a continuation.
    For example, let $K$ be $\langle K'; (1, 2, 3); v(abc) \\ ? \rangle$,
    then $K(v(ghi))$ is $\langle K'; (1, 2, 3); v(abc) \ v(ghi) \rangle$. A special
    continuation $halt$ terminates the abstract machine with provided result.

    \item $A$ is an accumulator, a sequence \ref{SequencesNotation} of already computed
    results.

    \item $C$ (control) is a concrete sequence \ref{ConcreteSequenceNotation} of terms
    upon which the abstract machine is operating right now. For example:
    \texttt{call(FOO, v(123) v(456)) v(w 8) v(blah)}.
\end{itemize}

And here are the computational rules:

\begin{figure}[H]
    \caption{Computational rules}

    \begin{align*}
        (v): \ & \langle K; A; v(\overline{tok}) \ t \ \overline{t'} \rangle & \to &
            \langle K; A, \ \overline{tok}; t \ \overline{t'} \rangle \\
        (v\mbox{-}end): \ & \langle K; A; v(\overline{tok}) \rangle & \to &
            K(unseq(A, \overline{tok})) \\
        (op): \ & \langle K; A; call(\overline{t}, \overline{a}) \ \overline{t'} \rangle & \to &
            \langle \langle K; A; call(?, \overline{a}) \ \overline{t'} \rangle; (); \overline{t} \rangle \\
        (args): \ & \langle K; A; call(ident, \overline{a}) \ \overline{t} \rangle & \to
            & \langle \langle K; A; ident(?) \ \overline{t} \rangle; (); comma\mbox{-}sep(\overline{a}) \rangle \\
        (start): \ & MACROLOP\_EVAL(t \ \overline{t'}) & \to &
            \langle halt; (); t \ \overline{t'} \rangle
    \end{align*}
\end{figure}

The following notations are used:

\begin{notation}[Reduction step]
    $\to$ denotes a single step of reduction (computation).
\end{notation}

\begin{notation}[Sequences]
    \label{SequencesNotation}
    \begin{enumerate}
        \item A sequence has the form $(x_1, \ldots, x_n)$.
        \item $()$ denotes the empty sequence.
        \item An element can be appended by comma: if $a = (1, 2, 3)$ and $b = 4$, then $a, b = (1, 2, 3, 4)$.
        \item \texttt{unseq} extracts elements from a sequence without a separator: \\
        \texttt{unseq((a, b, c)) = a b c}.
        \item \texttt{comma-sep} places \texttt{v(,)} between terms in a concrete sequence of terms:\\
        \texttt{comma-sep(v(123) call(FOO, v(a)) call(BAR, v(b)))} results in
        \texttt{v(123) v(,) call(FOO, v(a)) v(,) call(BAR, v(b))}.
    \end{enumerate}
\end{notation}

\begin{notation}[Concrete sequence]
    \label{ConcreteSequenceNotation}
    $\overline{x}$ denotes a concrete sequence $x_1 \ldots x_n$. For example:
    \texttt{v(abc) call(FOO, v(123)) v(u 8 9)}.
\end{notation}

\begin{notation}[Meta-variables]
    \ \\
    \begin{tabular}{|c|c|}
        \hline
        \texttt{tok} & C preprocessor token \\
        \texttt{ident} & C preprocessor identifier \\
        \texttt{t} & Macrolop term \\
        \texttt{a} & Macrolop term used as an argument \\
        \hline
    \end{tabular}
\end{notation}

Notes:

\begin{itemize}
    \item Look at $(args)$. Macrolop generates a usual C-style macro invocation with
    fully evaluated arguments, which will be then expanded by the C preprocessor, resulting
    in yet another concrete sequence of Macrolop terms to be evaluated by the computational
    rules. \par Therefore, an expansion of $call(\overline{t}, \overline{a})$
    must match the Macrolop grammar, otherwise it might result in a compilation error.
    \item With the current implementation, at most $2^{14}$ reduction steps are
    possible. After exceeding this limit, compilation will likely fail.
\end{itemize}

The rules are fairly simple: a concrete sequence of terms provided into \\
\texttt{MACROLOP\_EVAL} is evaluated sequentially till the end; a function's arguments
are evaluated before the function is applied, e.g. Macrolop follows applicative
evaluation strategy \cite{ApplicativeEvaluationStrategy}. When there's no more terms
to evaluate, the result is pasted where \texttt{MACROLOP\_EVAL} has been invoked.

The essence of the Macrolop metalanguage is that it allows recursive macro calls. But
to be precise, it allows only indirect recursion. First define the $\twoheadrightarrow$
meta-operator (resembles that in lambda calculi):

\begin{notation}[Multiple reduction steps]
    $\twoheadrightarrow$ denotes one or more single evaluation steps, e.g.
    $\overline{t} \twoheadrightarrow \overline{t'}$ is the same as
    $\overline{t} \to \ldots \to \overline{t'}$.
\end{notation}

Then consider these two cases:

\begin{itemize}
    \item Direct recursion: $call(X, \overline{tok}) \to \overline{t}$, where
    $\overline{t}$ contains $X$. Then this $X$ will be blocked forever due to the
    rules of the C preprocessor (an expansion of \texttt{X(...)} containing
    \texttt{X}).

    \item Indirect recursion: $call(X, \overline{a}) \twoheadrightarrow
    \overline{t} \ call(Y, \overline{a'}) \ \overline{t'}$ and
    $call(Y, \overline{a'}) \twoheadrightarrow \overline{t''}$, where $\overline{t''}$
    contains $X$. Then this $X$ will \textbf{not} be blocked by the C preprocessor,
    e.g. can be invoked again.
\end{itemize}

Now let's move to the examples of reduction. Take the following code:

\begin{minted}{c}
#define X(op)        call(op, v(123))
#define CALL_X(_123) call(X, v(ID))
#define ID(x)        v(x)
\end{minted}

See how \texttt{call(X, v(CALL\_X))} is evaluated:

\begin{example}[Evaluation of terms]
\small
\begin{gather*}
    MACROLOP\_EVAL(call(X, v(CALL\_X))) \\
    \downarrow (start) \\
    \langle halt; (); call(X, v(CALL\_X)) \rangle \\
    \downarrow (args) \\
    \langle \langle halt; (); X(?) \rangle; (); v(CALL\_X) \rangle \\
    \downarrow (v\mbox{-}end) \\
    \langle halt; (); call(CALL\_X, v(123)) \rangle \\
    \downarrow (args) \\
    \langle \langle halt; (); CALL\_X(?) \rangle; (); v(123) \rangle \\
    \downarrow (v\mbox{-}end) \\
    \langle halt; (); call(X, v(ID)) \rangle \\
    \downarrow (args) \\
    \langle \langle halt; (); X(?) \rangle; (); v(ID) \rangle \\
    \downarrow (v\mbox{-}end) \\
    \langle halt; (); call(ID, v(123)) \rangle \\
    \downarrow (args) \\
    \langle \langle halt; (); ID(?) \rangle; (); v(123) \rangle \\
    \downarrow (v\mbox{-}end) \\
    \langle halt; (); v(123) \rangle \\
    \downarrow (v\mbox{-}end) \\
    halt(123)
\end{gather*}
\normalsize
\end{example}

The analogous version written in ordinary C looks like this:

\begin{minted}{c}
#define X(op)        op(123)
#define CALL_X(_123) X(ID)
#define ID(x)        x
\end{minted}

However, unlike the Macrolop version above, it gets blocked due to the
second call to \texttt{X}:

$$
X(CALL\_X) \to CALL\_X(123) \to X(ID)
$$

\section{Caveats}

\begin{itemize}
\item Consider this scenario:
    \begin{itemize}
        \item You call \texttt{FOO(1, 2, 3)}
        \item It gets expanded by the C preprocessor (not by Macrolop)
        \item Its expansion contains \texttt{FOO}
    \end{itemize}
Then \texttt{FOO} gets blocked by the C preprocessor, e.g. Macrolop cannot handle ordinary
macro recursion; you must use \texttt{call} to be sure that recursive calls
will behave as expected.

I therefore recommend to use only primitive C-style macros (e.g. for performance
reasons or because of you cannot express them in Macrolop).
\end{itemize}

\emergencystretch=1em
\printbibliography

\end{document}
